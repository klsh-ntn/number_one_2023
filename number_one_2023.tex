% arara: xelatex: {shell: yes}
% %arara: biber
% %arara: xelatex: {shell: yes}
% %arara: xelatex: {shell: yes}

\documentclass[12pt]{article}

\usepackage{hyperref} % гиперссылки

\usepackage{tikz} % картинки в tikz
\usetikzlibrary{arrows.meta} % tikz-прибамбас для рисовки стрелочек подлиннее

\usepackage{microtype} % свешивание пунктуации

\usepackage{array} % для столбцов фиксированной ширины

\usepackage{indentfirst} % отступ в первом параграфе

\usepackage{sectsty} % для центрирования названий частей
\allsectionsfont{\centering}

\usepackage{amsmath} % куча стандартных математических плюшек
\usepackage{amssymb} % символы
\usepackage{amsthm} % теоремки

\usepackage{comment} % добавление длинных комментариев

\usepackage[top=2cm, left=1.2cm, right=1.2cm, bottom=2cm]{geometry} % размер текста на странице

\usepackage{lastpage} % чтобы узнать номер последней страницы

\usepackage{enumitem} % дополнительные плюшки для списков
%  например \begin{enumerate}[resume] позволяет продолжить нумерацию в новом списке

\usepackage{caption} % что-то делает с подписями рисунков :)

\usepackage{qcircuit} % для рисовки квантовых диаграмм
\usepackage{physics} % бракеты

\usepackage{answers} % разделение условий и ответов в упражнениях


\usepackage{fancyhdr} % весёлые колонтитулы
\pagestyle{fancy}
\lhead{Числа бывают\ldots{ } раз}
\chead{}
\rhead{КЛШ-2023 (46 сезон)}
\lfoot{}
\cfoot{}
\rfoot{\thepage/\pageref{LastPage}}
\renewcommand{\headrulewidth}{0.4pt}
\renewcommand{\footrulewidth}{0.4pt}



\usepackage{todonotes} % для вставки в документ заметок о том, что осталось сделать
% \todo{Здесь надо коэффициенты исправить}
% \missingfigure{Здесь будет Последний день Помпеи}
% \listoftodos — печатает все поставленные \todo'шки



\usepackage{booktabs} % красивые таблицы
% заповеди из докупентации:
% 1. Не используйте вертикальные линни
% 2. Не используйте двойные линии
% 3. Единицы измерения - в шапку таблицы
% 4. Не сокращайте .1 вместо 0.1
% 5. Повторяющееся значение повторяйте, а не говорите "то же"



\usepackage{fontspec} % что-то про шрифты?
\usepackage{polyglossia} % русификация xelatex

\setmainlanguage{russian}
\setotherlanguages{english}

% download "Linux Libertine" fonts:
% http://www.linuxlibertine.org/index.php?id=91&L=1
\setmainfont{Linux Libertine O} % or Helvetica, Arial, Cambria
% why do we need \newfontfamily:
% http://tex.stackexchange.com/questions/91507/
\newfontfamily{\cyrillicfonttt}{Linux Libertine O}

\AddEnumerateCounter{\asbuk}{\russian@alph}{щ} % для списков с русскими буквами
\setlist[enumerate, 2]{label=\asbuk*),ref=\asbuk*}

%% эконометрические сокращения
\DeclareMathOperator{\Cov}{Cov}
\DeclareMathOperator{\Arg}{Arg}
\DeclareMathOperator{\Corr}{Corr}
\DeclareMathOperator{\Var}{Var}
\DeclareMathOperator{\E}{\mathbb{E}}
\newcommand \hVar{\widehat{\Var}}
\newcommand \hCorr{\widehat{\Corr}}
\newcommand \hCov{\widehat{\Cov}}
\newcommand \cN{\mathcal{N}}
\let\P\relax
\DeclareMathOperator{\P}{\mathbb{P}}

\usepackage{multicol}

\usepackage[bibencoding = auto,
backend = biber,
sorting = none,
style=alphabetic]{biblatex}

\addbibresource{forecast_everything.bib}



% делаем короче интервал в списках
\setlength{\itemsep}{0pt}
\setlength{\parskip}{0pt}
\setlength{\parsep}{0pt}




\Newassociation{sol}{solution}{solution_file}
% sol — имя окружения внутри задач
% solution — имя окружения внутри solution_file
% solution_file — имя файла в который будет идти запись решений
% можно изменить далее по ходу
\Opensolutionfile{solution_file}[all_solutions]
% в квадратных скобках фактическое имя файла

% магия для автоматических гиперссылок задача-решение
\newlist{myenum}{enumerate}{3}
% \newcounter{problem}[chapter] % нумерация задач внутри глав
\newcounter{problem}[section]

\newenvironment{problem}%
{%
\refstepcounter{problem}%
%  hyperlink to solution
     \hypertarget{problem:{\thesection.\theproblem}}{} % нумерация внутри глав
     % \hypertarget{problem:{\theproblem}}{}
     \Writetofile{solution_file}{\protect\hypertarget{soln:\thesection.\theproblem}{}}
     %\Writetofile{solution_file}{\protect\hypertarget{soln:\theproblem}{}}
     \begin{myenum}[label=\bfseries\protect\hyperlink{soln:\thesection.\theproblem}{\thesection.\theproblem},ref=\thesection.\theproblem]
     % \begin{myenum}[label=\bfseries\protect\hyperlink{soln:\theproblem}{\theproblem},ref=\theproblem]
     \item%
    }%
    {%
    \end{myenum}}
% для гиперссылок обратно надо переопределять окружение
% это происходит непосредственно перед подключением файла с решениями



\theoremstyle{definition}
\newtheorem{definition}{Определение}



\begin{document}

\tableofcontents{}

\section*{Анонс}
...

\newpage
\setcounter{section}{0}

\section{Ним-сложение}

$\mathbb{Z}_{\geq 0}$ — целые неотрицательные числа.

$\oplus$ — ним-сложение: переводим число в двоичную систему счисления, складываем побитно без переноса (0+0=0, 1+0=1, 1+1=0), 
переводим обратно в исходную систему счисления. 

$\otimes$ — ним-умножение: переводим число в двоичную систему счисления, побитно умножаем, 
переводим в исходную систему счисления. 

\begin{enumerate}
  \item Найди $2 \oplus 2$, $10 \oplus 5$, $\underbrace{5 \oplus 5 \oplus \ldots \oplus 5}_{2023 \text{ раза}}$.
  \item Найди $2 \otimes 2$, $10 \otimes 5$, $\underbrace{5 \otimes 5 \otimes \ldots \otimes 5}_{2023 \text{ раза}}$.
  \item Всегда ли $a \oplus b$ = $b \oplus a$? 
  Придумай числа, нарушающие равенство, или объясни, почему равенство верно всегда. 
  \item Всегда ли $a \otimes b$ = $b \otimes a$? 
  Придумай числа, нарушающие равенство, или объясни, почему равенство верно всегда. 
  \item Реши уравнение $x \oplus 7 = 0$, $9 \oplus y = 11$. 
  \item Объясни, как устроено ним-вычитать числа? 
  Ним-вычитание должно быть обратным действием к ним-сложению.
  \item Придумай числа $a$, $b$ и $c$ такие, что $a\cdot (b\oplus c) = a\cdot b \oplus a\cdot c$.
  \item Придумай числа $a$, $b$ и $c$ такие, что $a\cdot (b\oplus c) \neq a\cdot b \oplus a\cdot c$.
  \item Реши уравнения $3\cdot x \oplus 12 = 0$ и $x \oplus x \oplus x \oplus 12 = 0$.
  \item В сумме $5 \oplus 10 \oplus 7$ замени одно из чисел на \textit{большее}, чтобы сумма превратилась в ноль.
  \item В сумме $5 \oplus 10 \oplus 7$ замени одно из чисел на \textit{меньшее}, чтобы сумма превратилась в ноль.
  \item Миша ним-складывает числа не превосходящие $10$, сколько максимум он может получить?
  \item Маша ним-складывает числа не превосходящие $7$, сколько максимум она может получить?
  \item Реши уравнение $x \otimes 3 = 6$ и $x \otimes 3 = 0$.
  \item Реши уравнение $x \otimes x \oplus 3 \otimes x \oplus 2 = 0$.
  \item Придумай числа $a$, $b$ и $c$ такие, что $a\otimes (b\oplus c) = a\otimes b \oplus a\otimes c$.
  \item Придумай числа $a$, $b$ и $c$ такие, что $a\otimes (b\oplus c) \neq a\otimes b \oplus a\otimes c$.
\end{enumerate}

\newpage
\setcounter{section}{0}

\section{Ним-сложение}

$\mathbb{Z}_{\geq 0}$ — целые неотрицательные числа.

$\oplus$ — ним-сложение: переводим число в двоичную систему счисления, складываем побитно без переноса (0+0=0, 1+0=1, 1+1=0), 
переводим обратно в исходную систему счисления. 

$\otimes$ — ним-умножение: переводим число в двоичную систему счисления, побитно умножаем, 
переводим в исходную систему счисления. 

\begin{enumerate}
  \item Найди $2 \oplus 2$, $10 \oplus 5$, $\underbrace{5 \oplus 5 \oplus \ldots \oplus 5}_{2023 \text{ раза}}$.
  \item Найди $2 \otimes 2$, $10 \otimes 5$, $\underbrace{5 \otimes 5 \otimes \ldots \otimes 5}_{2023 \text{ раза}}$.
  \item Всегда ли $a \oplus b$ = $b \oplus a$? 
  Придумай числа, нарушающие равенство, или объясни, почему равенство верно всегда. 
  \item Всегда ли $a \otimes b$ = $b \otimes a$? 
  Придумай числа, нарушающие равенство, или объясни, почему равенство верно всегда. 
  \item Реши уравнение $x \oplus 7 = 0$, $9 \oplus y = 11$. 
  \item Объясни, как устроено ним-вычитать числа? 
  Ним-вычитание должно быть обратным действием к ним-сложению.
  \item Придумай числа $a$, $b$ и $c$ такие, что $a\cdot (b\oplus c) = a\cdot b \oplus a\cdot c$.
  \item Придумай числа $a$, $b$ и $c$ такие, что $a\cdot (b\oplus c) \neq a\cdot b \oplus a\cdot c$.
  \item Реши уравнения $3\cdot x \oplus 12 = 0$ и $x \oplus x \oplus x \oplus 12 = 0$.
  \item В сумме $5 \oplus 10 \oplus 7$ замени одно из чисел на \textit{большее}, чтобы сумма превратилась в ноль.
  \item В сумме $5 \oplus 10 \oplus 7$ замени одно из чисел на \textit{меньшее}, чтобы сумма превратилась в ноль.
  \item Миша ним-складывает числа не превосходящие $10$, сколько максимум он может получить?
  \item Маша ним-складывает числа не превосходящие $7$, сколько максимум она может получить?
  \item Реши уравнение $x \otimes 3 = 6$ и $x \otimes 3 = 0$.
  \item Реши уравнение $x \otimes x \oplus 3 \otimes x \oplus 2 = 0$.
  \item Придумай числа $a$, $b$ и $c$ такие, что $a\otimes (b\oplus c) = a\otimes b \oplus a\otimes c$.
  \item Придумай числа $a$, $b$ и $c$ такие, что $a\otimes (b\oplus c) \neq a\otimes b \oplus a\otimes c$.
\end{enumerate}




\newcommand{\daytwo}{
\begin{enumerate}
  \item Классические правила игры Ним просты. Есть несколько кучек камней. За ход можно взять любое
  количество камней из одной кучки. Проигрывает тот, кто не может сделать ход.
  \begin{enumerate}
    \item Кто выигрывает, если имеется две кучи из 11 и 22 камней? Найди выигрышный ход.
    \item Кто выигрывает, если имеется 5 куч камней из 6, 7, 8, 9 и 10 камней? Найди выигрышный ход.
  \end{enumerate}
  \item Ним Ласкера. Есть несколько кучек камней. За ход разрешается: либо взять любое положительное количество
  камней из одной кучки, либо поделить любую кучку на две новые непустые кучки. Проигрывает
  тот, кто не может сделать ход.
  \begin{enumerate}
    \item Построй функцию Шпрага-Гранди для одной кучки из $n$ камней.
    \item Определи выигрышный ход в ситуации с тремя кучками из 2, 5 и 7 камней.
  \end{enumerate}
  \item Есть несколько кучек камней. За ход разрешается поделить любую кучку на две новые непустые кучки. 
  Проигрывает тот, кто не может сделать ход.
  \begin{enumerate}
    \item Построй функцию Шпрага-Гранди для одной кучки из $n$ камней.
    \item Определи выигрышный ход в ситуации с тремя кучками из 2, 5 и 7 камней.
  \end{enumerate}
  \item Кегли. В ряд стоят кегли. За ход разрешается выбить шаром одну или две рядом стоящие кегли. 
  Определи выигрышный ход для ряда из 8 кегель. 
  
\end{enumerate}
}
\newpage
\daytwo
\vfill
\daytwo
\newpage



\section{Лог. КЛШ-2023}

Курс выбрали 14 школьников.

\begin{enumerate}
  \item 
\end{enumerate}

В теховском файле \verb|\newpage| стоит, чтобы легко было скопировать секцию, для печати двух копий подряд на одном листе.
Это позволяет экономить бумагу и время при печати :)

\subsection{Плакат}





\Closesolutionfile{solution_file}

% для гиперссылок на условия
% http://tex.stackexchange.com/questions/45415
\renewenvironment{solution}[1]{%
         % add some glue
         \vskip .5cm plus 2cm minus 0.1cm%
         {\bfseries \hyperlink{problem:#1}{#1.}}%
}%
{%
}%



\section{Решения}
\input{all_solutions}


\section{Источники мудрости}

\todo[inline]{передалать потом в bib-файл}

\begin{enumerate}
\item 
\item 
\end{enumerate}

\printbibliography[heading=none]


\end{document}
